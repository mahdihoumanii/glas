\section{The Three-Phase Pipeline}\label{sec:pipeline}
%=============================================================================

The calculation proceeds through three main phases, each corresponding to a set of \textsc{GLAS} commands.

\subsection{Phase 1: Generation}

The \texttt{generate} command initiates a new calculation:
\begin{lstlisting}
glas> generate g g > t t~ --jobs 8
\end{lstlisting}

This command:
\begin{enumerate}
    \item Creates a new run directory with a unique identifier.
    \item Invokes \textsc{QGRAF} to generate tree-level ($\ell=0$) and one-loop ($\ell=1$) diagrams.
    \item Parses the \textsc{QGRAF} output and stores diagram counts in metadata.
    \item Prepares the \textsc{FORM} project structure with procedures and include files.
\end{enumerate}

The process specification follows the standard notation: initial-state particles on the left of ``\texttt{>}'' and final-state particles on the right. Antiparticles are denoted by a tilde (e.g., \texttt{t\~{}} for $\bar{t}$).

\subsection{Phase 2: Evaluation}

The evaluation phase applies Feynman rules and performs algebraic simplifications:
\begin{lstlisting}
glas> evaluate lo --jobs 8           # Tree-level
glas> evaluate nlo --jobs 8 --dirac  # One-loop with Dirac simplification
\end{lstlisting}

For each diagram, the \textsc{FORM} driver:
\begin{enumerate}
    \item Includes the diagram expression from \textsc{QGRAF} output.
    \item Applies Feynman rules via \texttt{\#call FeynmanRules}.
    \item Substitutes Mandelstam invariants.
    \item Optionally performs Dirac algebra simplification.
    \item Writes the result to \texttt{Files/Amps/amp\{0,1\}l/d\{i\}.h}.
\end{enumerate}

The \texttt{--dirac} flag enables simultaneous Dirac simplification, which includes:
\begin{itemize}
    \item Gamma matrix algebra using the Chisholm identity.
    \item Trace evaluation for closed fermion loops.
    \item Spinor orthogonality conditions for external fermions.
\end{itemize}

\subsection{Phase 3: Contraction}

The contraction phase squares amplitudes and sums over helicities:
\begin{lstlisting}
glas> contract lo --jobs 4   # |M_0|^2
glas> contract nlo --jobs 4  # Re(M_0^* M_1)
\end{lstlisting}

For LO contractions, we compute $|\mathcal{M}_0|^2$ summed over colors and helicities. For NLO, we compute the interference term $\Re(\mathcal{M}_0^* \mathcal{M}_1)$.

The contraction procedure includes:
\begin{enumerate}
    \item Complex conjugation of amplitudes.
    \item Polarization sum insertion for external gluons.
    \item Color algebra evaluation.
    \item Combination of diagram pairs.
\end{enumerate}

%=============================================================================
