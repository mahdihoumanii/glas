\section{Introduction}\label{sec:intro}
%=============================================================================

Precision predictions for hadron collider observables require the computation of scattering amplitudes at next-to-leading order (NLO) and beyond in the strong coupling expansion. While significant progress has been achieved in automating such calculations~\cite{Signer:2008va,Frixione:2007vw,Alwall:2014hca}, the workflow remains technically demanding, involving multiple specialized tools that must be orchestrated carefully.

The typical NLO calculation pipeline consists of several stages:
\begin{enumerate}
    \item \textbf{Diagram generation}: Enumeration of all Feynman diagrams contributing at a given loop order.
    \item \textbf{Amplitude construction}: Application of Feynman rules to obtain algebraic expressions.
    \item \textbf{Amplitude manipulation}: Dirac algebra, color algebra, and kinematic simplifications.
    \item \textbf{Integral identification}: Extraction of loop integral topologies from the amplitude.
    \item \textbf{Integral reduction}: Reduction to master integrals via integration-by-parts (IBP) identities.
    \item \textbf{Renormalization}: Renormalization of UV and IR terms.
    \item \textbf{Master integral evaluation}: Numerical or analytical computation of the master integrals.
\end{enumerate}

\textsc{GLAS} addresses stages 1--7 by providing a unified command-line interface that automates the workflow and manages intermediate results. The system is designed with the following principles:
\begin{itemize}
    \item \textbf{Modularity}: Each stage is implemented as a separate module with well-defined interfaces.
    \item \textbf{Parallelism}: Computationally intensive tasks are distributed across multiple cores.
    \item \textbf{Reproducibility}: All intermediate results are stored in structured run directories.
    \item \textbf{Extensibility}: New processes and models can be added through configuration files.
\end{itemize}

This report is organized as follows. In Section~\ref{sec:architecture} we describe the overall architecture and run-based workflow. Section~\ref{sec:pipeline} details the three-phase calculation pipeline. The topology extraction algorithm is presented in Section~\ref{sec:topology}, followed by the IBP reduction procedure in Section~\ref{sec:ibp}. Section~\ref{sec:renormalization} discusses UV renormalization and counterterm computation. The finite field approach for linear relation identification is detailed in Section~\ref{sec:linrels}. We discuss the command interface in Section~\ref{sec:commands} and conclude in Section~\ref{sec:conclusions}.

Glas can be installed by cloning the git repository 
\begin{verbatim}
                git clone https://github.com/mahdihoumanii/glas.git
\end{verbatim}
A video for how to use glas is given in this \href{https://drive.google.com/file/d/1xFFeFzAeUpV_n_t7iYSmw0U3TrtDlvrC/view?usp=share_link}{link}.

%=============================================================================
