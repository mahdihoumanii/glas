\section{IBP Reduction}\label{sec:ibp}
%=============================================================================

The IBP reduction stage reduces all loop integrals to a minimal set of master integrals using integration-by-parts identities~\cite{Chetyrkin:1981qh,Tkachov:1981wb}.

\subsection{Mandelstam Preparation}

The script \texttt{mandIBP.m} computes the kinematic replacements needed for the reduction, storing results in \texttt{Files/mands.m}.
For 5 particle processes the Mandelstam variables are the set: 
\begin{equation}
    \{s_{12},s_{23},s_{34},s_{45},s_{15}\},
\end{equation}

and for 4 particle processes the Mandelstam variables are the set: 
\begin{align}
    &\{s_{12},s_{13},s_{14}\}\quad \text{where }\\
    &s_{14} = \sum_i^4 m_i^2 - s_{12}-s_{13},
\end{align}
and the mandelstam variables are defined as: 
\begin{align}
    s_{ij} &= (p_i + p_j) \quad \text{if both momenta are incoming/outgoing} \\
    s_{ij} &= (p_i - p_j) \quad \text{if one of momenta is incoming and the other one is outgoing} 
\end{align}
with on-shell external momenta $p_i\cdot p_i=m_i^2$.
\subsection{IBP Reduction with Blade}

The main reduction is performed by \texttt{IBP.m} using the \textsc{Blade} package. For each topology $T_i$, the reduction produces:
\begin{equation}
    I_{T_i}[\{n_1,n_2,\ldots\}] = \sum_j c_j(\epsilon, s_{ij}, m^2) \, M_j
\end{equation}
where $M_j$ are master integrals and $c_j$ are rational functions of the kinematic invariants and the dimensional regulator $\epsilon = (4-D)/2$.

The output includes:
\begin{itemize}
    \item \texttt{Files/IBP/IBP\{i\}.m}: Mathematica-format reduction rules.
    \item \texttt{form/Files/IBP/IBP\{i\}.h}: \textsc{FORM}-format reduction rules.
\end{itemize}
This workflow targets massive five-point amplitudes. IBP reduction can be expensive, so the top-quark mass is set to one to accelerate finite-field reconstruction of the IBP relations. After the reduction, the mass is restored by dimensional analysis (see the \texttt{RestoreMass} function in \texttt{glas/mathematica/scripts/IBP.m}). The result is then partially fractioned using \texttt{MultivariateApart}; \texttt{MultivariatePassToSingular} computes a Groebner basis in \texttt{Singular} and performs the reduction in \texttt{FORM}.

\subsection{Symmetry Relations and PaVe Conversion}

The script \texttt{SymmetryRelations.m} identifies relations between master integrals from different topologies and converts to the standard Passarino-Veltman (PaVe) basis~\cite{Passarino:1978jh}. The output includes:
\begin{itemize}
    \item \texttt{Files/SymmetryRelations.m}: Master integral list and PaVe rules.
    \item \texttt{form/Files/SymmetryRelations.h}: \textsc{FORM} substitution rules.
    \item \texttt{form/Files/MastersToSym.h}: Symbolic master integral identifiers.
\end{itemize}


This pipeline is executed with:
\begin{lstlisting}
glas> ibp
\end{lstlisting}

\subsection{Master Integral Coefficient Extraction}

The \texttt{reduce} command applies the IBP reduction rules via:
\begin{lstlisting}
glas> reduce --jobs 4
\end{lstlisting}

The \texttt{reduce} command applies the IBP relations to the contracted amplitude and saves the results in:
\begin{itemize}
    \item \texttt{Mathematica/Files/M0M1Reduced/}
\end{itemize}
The master-integral coefficients are then extracted with:

\begin{lstlisting}
glas> micoef --jobs 4 [--combine]
\end{lstlisting}

This writes the coefficients of the master integrals as:
\begin{itemize}
    \item \texttt{form/Files/MasterCoefs/c\{i\}.h}: Coefficient of master integral $i$.
    \item \texttt{Mathematica/Files/MasterCoefficients/mi\{i\}/MasterCoefficient\{i\}.m}: Per-master coefficients and independent-basis rules.
    \item \texttt{Mathematica/Files/MasterCoefficients.m}: Combined coefficient file if the \texttt{--combine} flag is used.
\end{itemize}
