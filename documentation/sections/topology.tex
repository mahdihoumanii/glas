\section{Topology Extraction}\label{sec:topology}
%=============================================================================

After contraction, the NLO amplitude contains scalar Feynman integrals that must be identified and reduced. The topology extraction proceeds in four stages.

\subsection{Stage 1: Incomplete Topology Identification}

The Mathematica script \texttt{extract\_topologies\_stage1.m} loads the contracted amplitudes and identifies the propagator structures. Each loop integral is characterized by its set of propagators:
\begin{equation}
    I[\{q_1^2-m_1^2, q_2^2-m_2^2, \ldots\}] = \int \frac{\mathrm{d}^D \ell}{(2\pi)^D} \frac{1}{(q_1^2-m_1^2)(q_2^2-m_2^2)\cdots}
\end{equation}
where the $q_i$ are linear combinations of the loop momentum $\ell$ and external momenta. This step is done in \texttt{FeynCalc} to extract and identify topologies.

The output is a list of ``incomplete'' topologies---propagator sets that may not span the full space needed for IBP reduction.

\subsection{Stage 2: Topology Completion}

The Python script \texttt{extend.py} completes each topology by adding auxiliary propagators. For a one-loop amplitude with $n$ external legs, a complete topology requires $n+1$ propagators (one more than the number of independent scalar products).

The completion algorithm uses breadth-first search (BFS) on an integer lattice where each node represents a shift vector for the propagator momentum:
\begin{equation}
    q = \ell + \sum_{i=1}^{n-1} a_i p_i, \quad a_i \in \mathbb{Z}
\end{equation}

The algorithm prioritizes:
\begin{enumerate}
    \item Filling gaps between existing propagators if the path includes missing momentum directions.
    \item Adding propagators from endpoints in directions not yet covered.
\end{enumerate}

This ensures that the completed topology has propagators depending on all external momentum directions, which is essential for a valid IBP reduction.

\subsection{Stage 2b: Topology Mapping}

The script \texttt{extract\_topologies\_stage2.m} maps the amplitude integrals onto the completed topologies using \textsc{FeynCalc}'s \texttt{FCLoopFindTopologies} and related functions. The output includes:
\begin{itemize}
    \item \texttt{Files/integrals.m}: Integral definitions in Mathematica format.
    \item \texttt{form/Files/intrule.h}: \textsc{FORM}-formatted substitution rules.
\end{itemize}

\subsection{Stage 3: Parallel FORM Processing}

The final topology extraction stage processes the mapped integrals in parallel using \textsc{FORM}. The \texttt{ToTopos\_J\{k\}of\{N\}.frm} drivers:
\begin{enumerate}
    \item Load the integral rules from \texttt{intrule.h}.
    \item Apply substitutions to express all integrals in terms of standard topology notation.
    \item Write output to \texttt{Files/M0M1top/d\{i\}x\{j\}.h}.
\end{enumerate}
This stage is executed by the command:
\begin{lstlisting}
glas> extract topologies
\end{lstlisting}
%=============================================================================
