\section{Conclusions}
\label{sec:conclusions}
%=============================================================================

We have presented \textsc{GLAS}, a unified framework for automated NLO QCD calculations. The system integrates diagram generation, symbolic manipulation, integral reduction, renormalization, and coefficient simplification into a coherent pipeline with parallel execution capabilities.

Key features include:
\begin{itemize}
    \item Run-based workflow with complete reproducibility.
    \item Parallel execution at the diagram level.
    \item Automated topology completion and IBP reduction.
    \item $\overline{\text{MS}}$ renormalization with UV counterterm computation.
    \item Finite field techniques for linear relation identification via \textsc{FiniteFlow}.
    \item Verbose streaming mode for debugging and monitoring.
    \item Modular architecture enabling easy extension.
\end{itemize}

The finite field approach implemented in the \texttt{linrels} command provides significant computational advantages for complex processes, enabling the simplification of master integral coefficients that would be intractable with purely symbolic methods.

Future developments will include:
\begin{enumerate}
    \item Support for full finite field reconstruction for one-loop helicity amplitude in the t'HV scheme.
    \item Faster topology extraction.
    \item Higgs processes 
    \item Kira integration
    \item $\epsilon$ factorized system of differential equations to solve the Master Integrals up to $\mathcal{O}(\epsilon^2)$
\end{enumerate}


%=============================================================================
